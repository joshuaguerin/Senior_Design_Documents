\documentclass[xcolor=table]{beamer}

\usetheme[secheader,compress]{Madrid} %Primary theme

\usepackage{verbatim}
\usepackage{graphicx}

%%%%%%%%%%% BEGIN MACROS %%%%%%%%%%%%%%%%%%
% frameT: Frame with title
\newcommand{\frameT}[2]{\frame{\frametitle{#1} #2}}

% frameF: Fragile frame with title
\newcommand{\frameF}[2]{
  \begin{frame}[fragile]
    \frametitle{#1}
    #2
  \end{frame}
}

% frameTop: Frame aligned t the top
\newcommand{\frameTop}[2]{\frame[t]{\frametitle{#1} #2}}


\newcommand{\tab}{\hspace{1cm}}

\newcommand{\spaceor}{\hspace{5pt} \textbf{or} \hspace{5pt}}

%%%%%%%%%%% END MACROS %%%%%%%%%%%%%%%%%%%%



\begin{document}

\title{Title here!}

\author{Author list here!}
\institute{UT-Martin}
\date{\today}

%%%%%%%%%%% BEGIN TITLE %%%%%%%%%%%%%%%%%%
\frame{\titlepage}

 %\section{Outline}
%%%%%%%%%%%% END TITLE  %%%%%%%%%%%%%%%%%%


\section{Introduction}
\frameT{Motivation} {
  List some of your motivations here!
  \bigskip
  \begin{enumerate}
    \item Some ideas
      \bigskip
    \item More ideas!
  \end{enumerate}

  \bigskip
  
  One or more slides of background or motivation.
}



\section{Sections--a useful organizational tool.}

%% \frameT{Logic Preliminaries} {
%%   \begin{itemize}
%%     \item Conjunction
%%       \begin{center}
%%         $blond(mike) \land tall(mike)$ \tab ``mike is blond AND mike is tall''
%%       \end{center}
%%       \bigskip

%%     %%\item $\lor$ \tab Disjunction
%%     \item Disjunction
%%       \begin{center}
%%         $blank \lor blank$ \tab ``bland \textit{OR} blank''
%%       \end{center}
%%       \bigskip 

%%     %% \item $\rightarrow$ \tab Implication
%%     \item Implication
%%       \begin{center}
%%         $cat(tom) \rightarrow mammal(tom)$ \tab ``If tom is a cat THEN tom is a mammal'' %or ``blank implies blank''
%%       \end{center}
%%       \bigskip

%%     \item Negation
%%     %% \item $\lnot$ \tab Negation
%%       \begin{center}
%%         $\lnot blank$ \tab ``Not blank''
%%       \end{center}

%%   \end{itemize}
%% }




\begin{frame}[fragile]
\frametitle{Family Tree Knowledge Base}
Facts:
\begin{verbatim}
Verbatim is a great way of enumerating code/algorithmic ideas.
\end{verbatim}
\end{frame}


\frameT{How to include images} {
  %% \includegraphics[width=.7\linewidth]{figures/image.pdf}
}


\begin{frame}[fragile]
  \frametitle{Social Network Graph}
  \begin{figure}[ht]
    \begin{minipage}[b]{0.53\linewidth}
      \centering
      Minipages are a great way to
    \end{minipage}
    \hspace{0.5cm}
    \begin{minipage}[b]{0.4\linewidth}
      \centering
      Line up side-by-side content.

    \end{minipage}
  \end{figure}
  
\end{frame}





\frameT{Any Questions?} {
  It is a good idea to have a page inviting the audience to interact. You may learn quite a bit from each other.
  \begin{center}
    Questions?
  \end{center}
  \begin{center}
    Comments?
  \end{center}
}

%\frameF{fragile test} {
%}

%% \frameF{Prolog Family Tree} {
%% \begin{verbatim}
%% hello
%% \end{verbatim}



%% }

%Empty Page
%\frameT{Frame 1}{
%}  


\end{document}
